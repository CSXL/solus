\documentclass[a4paper,twocolumn,11pt]{quantumarticle}
\pdfoutput=1
\usepackage[utf8]{inputenc}
\usepackage[english]{babel}
\usepackage[T1]{fontenc}
\usepackage{amsmath}
\usepackage{hyperref}

\usepackage{tikz}
\usepackage{lipsum}

\usepackage{natbib}
\bibliographystyle{unsrt}

\newtheorem{theorem}{Theorem}
\begin{document}

\title{Solus: An end-to-end AI software developer.}

\author{Adam Blumenfeld}
\affiliation{President, CSX Labs}
\email{adamb@csxlabs.org}
\orcid{0000-0001-9995-3327}
\maketitle

\begin{abstract}
  TODO: Fill in abstract.
\end{abstract}
\section{Introduction}
\subsection{Market Background}
The seamless transfer of knowledge and data across nations powered the information age. Global access to electronic devices created high-growth platforms and services, springing millions of jobs in content creation, software development, and more. An estimated 26.9 million professional developers practiced worldwide as of 2022, growing 14.5\% from 2018\cite{Qubit2022How}. The inception of large language models gave rise to the generative AI space, with platforms growing at a record pace, such as OpenAI's ChatGPT, which rose to 100 million users in two months\cite{Carr2023ChatGPT}. One implication of generative AI is that it automates large portions of the jobs created in the information revolution in the tech, legal, financial, media, and support fields, to name a few\cite{Mok2023ChatGPT}. S\&P Global Market Intelligence predicts generative AI market revenue to hit \$36 billion by 2028, forecasting code generators to lead the expansion at a compound annual growth rate of 72.9\%\cite{Park2023Generative}. The ability of language models to generate code has sparked new products such as GitHub's Copilot\cite{Dohmke2022GitHub}, Replit's Ghostwriter\cite{Masad2022Ghostwriter}, and Amazon's Codewhisperer\cite{Engdahl2023Amazon}, which provide inline code completions saving developers time from repetitive tasks, routine lookups, and forgotten implementations.

Inline completions are helpful but far from autonomous, requiring manual developer intervention and revision.

Projects like Smol\cite{Osika2023Current}, AutoGPT\cite{Ortiz2023What}, and BabyAGI\cite{Parthasarathy2023Meet} aim to perform tasks autonomously by making language models reason and evaluate themselves. However, projects like AutoGPT and BabyAGI are general in project scope, leading to a jack-of-all-trades issue.

In contrast, Smol is simple though it lacks capability and requires much human intervention.

\subsection{Problems}
\subsubsection{Scope Limits Capability}
Tools like AutoGPT and BabyAGI aim to achieve artificial general intelligence (AGI), a system that can learn to carry out any task a human or animal can perform\cite{Bubeck2023Sparks}. Although a novel goal, refraining from tailoring the generative process to a specific task limits the value, the systems can bring to such a complex task as software development. A capable AI software developer project must tailor agents to the software development lifecycle.

\subsubsection{Multi-Agent Collaboration}
There are no effective frameworks for multi-agent communication and
collaboration on a resource in an organized manner. Software development is an
inherently complex process, with many scoped concerns spanning a project with
many dependencies and business-related tradeoffs to weigh in. As agents act on a
project's resources, a standard protocol must exist for compartmentalizing
working areas, communicating issues, and resolving conflicts.

\subsubsection{Manual Feedback Loop}
Most tools require an amount of human feedback and approval when operating to
stay on track and approve resource usage. Language model costs incurred during
critical reasoning and self-regulation limit more capable projects, such as
AutoGPT. These costs are a barrier to making multi-agent systems possible, as the
quality of these tools' performance as software developers render the cost
unjustified. Therefore, an AI software developer system must have transparent
resource monitoring and self-guidance, ensuring the quality of iteration.

\subsection{Solution}
Solus \textit{will be}\footnote{NOTE: We are raising funding to build Solus due to the steep upstart costs. Contact the corresponding author if you want to contribute financially.} an end-to-end AI software development solution utilizing self-critical, multi-role AI agents.

\subsubsection{System Requirements}
Solus must be robust, secure, and scalable to be a suitable system for enterprises to trust with their projects.

We achieve this by compartmentalizing agents and adding layers of abstraction to inter-agent communication, allowing us to optimize, shard, and secure resources under the hood.

\subsubsection{Product Requirements}
The system must be capable of doing most software development tasks, such as managing dependencies, refactoring, debugging, and generating business logic and documentation related to the project and business goals. It must execute these tasks self-regulating while ensuring all operations are transparent, traceable, and intervenable. We accomplish this by orchestrating the agents in a service mesh where sidecars intercept communications between agents and resources and beam them to a centralized control plane that controls downstream processes and provides visibility over the operation.

\subsection{Business Requirements}
Solus must be versatile, profitable, and sustainable. By making agents customizable components with a clear separation of concerns, we can give Solus a clear path of expansion to other applications in the generative AI space. By operating on a cloud model, we can stay profitable on a low operating margin minimizing costs for us and our customers. To make Solus sustainable, we plan on creating a plugin ecosystem for optimized agent types for specific tasks and technologies. These developer ecosystems give Solus an economic moat over other initiatives. We also plan on building Solus to be open-source to ensure the most significant impact on the developer community. Open source allows us to create an ecosystem around our technology while maintaining profit and advantage due to enterprise trust in our platform.

\bibliography{whitepaper}
\end{document}