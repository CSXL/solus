\documentclass[a4paper,twocolumn,11pt,accepted=2017-05-09]{quantumarticle}
\pdfoutput=1
\usepackage[utf8]{inputenc}
\usepackage[english]{babel}
\usepackage[T1]{fontenc}
\usepackage{amsmath}
\usepackage{hyperref}

\usepackage{tikz}
\usepackage{lipsum}

\usepackage{natbib}
\bibliographystyle{unsrt}

\newtheorem{theorem}{Theorem}

\begin{document}

\title{Solus: An end-to-end AI software developer.}

\author{Adam Blumenfeld}
\affiliation{President, CSX Labs}
\email{adamb@csxlabs.org}
\orcid{0000-0001-9995-3327}
\maketitle

\begin{abstract}
  CSX Labs
\end{abstract}
\section{Introduction}
\subsection{Market Background}
The seamless transfer of knowledge and data across nations powered the information age. Global access to electronic devices created high-growth platforms and services, springing millions of jobs in content creation, software development, and more. An estimated 26.9 million professional developers practiced worldwide as of 2022, growing 14.5\% from 2018\cite{Qubit2022How}. The inception of large language models gave rise to the generative AI space, with platforms growing at a record pace, such as OpenAI's ChatGPT, which rose to 100 million users in two months\cite{Carr2023ChatGPT}. One implication of generative AI is that it automates large portions of the jobs created in the information revolution in the tech, legal, financial, media, and support fields, to name a few\cite{Mok2023ChatGPT}. S\&P Global Market Intelligence predicts generative AI market revenue to hit \$36 billion by 2028, forecasting code generators to lead the expansion at a compound annual growth rate of 72.9\%\cite{Park2023Generative}. The ability of language models to generate code has sparked new products such as GitHub's Copilot\cite{Dohmke2022GitHub}, Replit's Ghostwriter\cite{Masad2022Ghostwriter}, and Amazon's Codewhisperer\cite{Engdahl2023Amazon}, which provide inline code completions saving developers time from repetitive tasks, routine lookups, and forgotten implementations.

Inline completions are helpful but far from autonomous, requiring manual developer intervention and revision.

Projects like Smol\cite{Osika2023Current}, AutoGPT\cite{Ortiz2023What}, and BabyAGI\cite{Parthasarathy2023Meet} aim to perform tasks autonomously by making language models reason and evaluate themselves. However, projects like AutoGPT and BabyAGI are general in project scope, leading to a jack-of-all-trades issue.

In contrast, Smol is simple though it lacks capability and requires much human intervention.

\bibliography{whitepaper}
\end{document}